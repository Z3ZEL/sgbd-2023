\documentclass[a4paper,11pt]{article} % A4 paper size and default 11pt font size

\usepackage{geometry}
\geometry{margin=2cm}

\usepackage[utf8]{inputenc} % Required for inputting international characters
\usepackage[T1]{fontenc} % Output font encoding for international characters

\usepackage[french]{babel} % Support for French language
\usepackage{lmodern} % Use a modern font
\usepackage{graphicx} % Required for inserting images
\usepackage{enumitem} % For customizing list spacing

\title{Garage - Modèle relationnel}
\author{Côme Perin \\
        Mathis Reinert \\
        Esteban Rodriguez \\
        Antoine L'Honoré}
\date{Octobre 2023}

\begin{document}

\maketitle

\section{Introduction}
Ce document présente le modèle relationnel mis en place lors de la réalisation de la base de donnée du projet de SGBD.

\section{Modèle relationnel}
\fontsize{9pt}{10.8pt}\selectfont

\begin{itemize}[leftmargin=*]
    \item \texttt{adresses( \\
        \underline{numero\_adresse}, \\
        numero\_habitation, \\
        nom\_voie, \\
        code\_postal, \\
        pays \\
    )}
    \item \texttt{clients( \\
        \underline{numero\_client}, \\
        \#numero\_adresse, \\
        nom\_client, \\
        prenom\_client, \\
        mail\_client, \\
        telephone\_client \\
    )}
    \item \texttt{voitures( \\
        \underline{matricule\_voiture}, \\
        \#numero\_modele, \\
        \#numero\_client,    \\
        puissance\_fiscale\_voiture, \\
        cylindree\_voiture, \\
        couleur\_voiture, \\
        annee\_circulation\_voiture \\
    )}
    \item \texttt{modeles\_voitures( \\
        \underline{numero\_modele}, \\
        type\_modele, \\
        marque\_modele, \\
        version\_modele, \\
        annee\_modele \\
    )}
    \item \texttt{composition( \\
        \underline{\#numero\_modele, \#numero\_serie\_piece} \\
    )}
    \item \texttt{pieces( \\
        \underline{numero\_serie\_piece}, \\
        type\_piece, \\
        prix\_piece, \\
        marque\_piece \\
    )}
    \item \texttt{requerir( \\
        \underline{\#numero\_action, \#numero\_serie\_piece}, \\
        quantite \\
    )}
    \item \texttt{actions( \\
        \underline{numero\_action}, \\
        nom\_action, \\
        duree\_estime\_action, \\
        tarif\_action \\
    )}
    \item \texttt{contenir( \\
        \underline{\#numero\_action, \#numero\_intervention} \\
    )}
    \item \texttt{survenir( \\
        \underline{\#numero\_action, \#numero\_intervention} \\
    )}
    \item \texttt{factures( \\
        \underline{numero\_facture}, \\
        \#numero\_intervention, \\
        date\_facture, \\
        montant\_facture \\
    )}
    \item \texttt{devis( \\
        \underline{numero\_devis}, \\
        \#numero\_intervention, \\
        date\_devis, \\
        montant\_devis \\
    )}
    \item \texttt{interventions( \\
        \underline{numero\_intervention}, \\
        \#matricule\_voiture, \\
        \#numero\_SIREN, \\
        date\_debut\_intervention, \\
        date\_fin\_intervention, \\
        kilometrage \\
    )}
    \item \texttt{garages( \\
        \underline{numero\_SIREN}, \\
        \#numero\_adresse, \\
        denomination\_garage, \\
        forme\_juridique\_garage, \\
        date\_creation\_garage \\
    )}
    \item \texttt{employes( \\
        \underline{numero\_securite\_sociale}, \\
        \#numero\_SIREN, \\
        \#numero\_adresse, \\
        nom\_employe, \\
        prenom\_employe, \\
        date\_embauche\_employe, \\
        salaire\_horaire\_employe \\
    )}
\end{itemize}

\end{document}
